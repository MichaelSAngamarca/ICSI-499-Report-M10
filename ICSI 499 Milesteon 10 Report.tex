\documentclass[12pt]{article}

\usepackage[margin=1in]{geometry}
\usepackage{setspace}
\usepackage{graphicx}
\usepackage{tikz}

\begin{document}

% ----------------- TITLE PAGE -----------------
\begin{titlepage}
    \thispagestyle{empty}
    \centering
    \vspace*{2cm}

    {\large ICSI 499 Capstone Project Report \par}
    \vspace{1.5cm}

    {\Huge \textbf{TalkAssist} \par}
    \vspace{2cm}

    {\large \textit{Team 5} \par}
    \vspace{0.5cm}

    College of Nanotechnology, Science, and Engineering \\
    University at Albany, SUNY

    \vspace{3cm}

    {\large \textit{Project Sponsor:} \par}
    \vspace{0.5cm}

    Dr. Pradeep Atrey \\
    Athulya Mathew \\
    Department of Computer Science \\
    College of Nanotechnology, Science, Engineering \\
    University at Albany \\
    UAB 421, 1215 Western Avenue, Albany, NY 12222

    \vspace{3cm}

    {\large \today}
\end{titlepage}

% ----------------- FRONT MATTER -----------------
\pagenumbering{roman}

% ----------------- ACKNOWLEDGMENTS -----------------
\newpage
\thispagestyle{plain}
\begin{center}
    \Large \textbf{Acknowledgments}
\end{center}

\vspace{1cm}

\noindent
Team 5 would like to extend our heartfelt appreciation to Professor Pradeep Atrey and Athuyla Matthew for their unwavering support and encouragement. Their guidance throughout the development of TalkAssist inspired us, challenged us, and enabled us to transform our vision into a working product.

\vspace{0.5cm}

\noindent
The motivation behind Talk Assist was to develop a tool that people with life-altering visual disabilities might utilize to live their lives on an equal footing with people without them. We aim to sustain this purpose and produce a product that satisfies these great expectations placed upon us by our sponsors and the University at Albany.

% ----------------- ABSTRACT -----------------
\newpage
\thispagestyle{plain}
\begin{center}
    \Large \textbf{Abstract}
\end{center}

\vspace{1cm}

\noindent
Talk Assist is an artificial intelligent voice assistant designed for individuals with visual disabilities. The project's objective is to create, test, and assess an AI-driven voice assistant that supports visually impaired users in managing daily life. Talk Assist includes two unique modes that allow users to dynamically transition based on Internet connectivity. Users can request online searches, weather, time, date information, and reminders in Online Mode. In Offline Mode, Talk Assist preserves the same weather, time, date, and reminder functionality, while also offering a calculation utility for basic computations. Talk Assist aims to simplify daily tasks for visually impaired users without the need for expensive or inaccessible healthcare technologies.

% ----------------- PROBLEM ANALYSIS -----------------
\newpage
\pagenumbering{arabic}

\section{Problem Analysis}

\subsection{Project Statement}
Individuals with visual impairments often face significant challenges when interacting with everyday technology. Tasks such as reading screens, managing schedules, accessing online information, or performing basic device operations can require specialized tools or expensive accessibility technologies. Many existing solutions lack adaptability, rely heavily on constant internet connectivity, or are too complex for seamless daily use.

TalkAssist aims to address these challenges by providing an intuitive, AI powered voice assistant specifically designed for users with limited or no vision. The goal of this project is to design, implement, and evaluate a system that supports both online and offline functionality, delivers accurate voice feedback, and simplifies daily tasks through natural speech interaction. By integrating reliable recognition, high quality text-to-speech components, and an interface tailored for accessibility, TalkAssist seeks to make digital interaction more equitable and practical for visually impaired individuals.

\subsection{What are Existing Solutions?}
Several solutions exist in the accessibility and voice-assistant space, such as Apple’s VoiceOver, Google Assistant, and Amazon Alexa. While powerful, these tools are developed for broad consumer audiences rather than tailored explicitly for visually impaired users. As a result, they often require complex setups, consistent cloud connectivity, or rely on visual interfaces not optimized for non-sighted users.

Some standalone accessibility devices provide improved usability such as scre, but they can be costly and inaccessible to many. These systems also frequently depend on external hardware or lack offline capability, making them unreliable in environments with unstable internet connections.

Additionally, visually impaired users report issues such as inconsistent voice feedback, poor environmental adaptability, and difficulty performing basic tasks when offline. Users often struggle to transition smoothly between connected and disconnected settings, resulting in a fragmented user experience.

\subsection{Our Solution}
TalkAssist provides a unified, accessibility focused platform designed to meet the needs of visually impaired users in both online and offline contexts. Our system introduces the following key features:

\begin{itemize}
    \item \textbf{Dual-Mode Functionality:} Users can operate TalkAssist in Online Mode for tasks such as web searches, weather updates, and general questions, while Offline Mode preserves essential functions like time, date, reminders, and basic calculations.
    
    \item \textbf{Accessible Voice Interaction:} The system is designed around natural speech input and clear spoken output, minimizing dependence on visual cues or screen navigation.
    
    \item \textbf{Task Management and Reminders:} Users can create, update, and manage reminders through simple voice commands in both online and offline modes, supporting daily independence.
    
    \item \textbf{Lightweight and Low-Cost Design:} Unlike expensive commercial assistive devices, TalkAssist runs on standard consumer hardware, making it affordable and widely accessible.
\end{itemize}

Together, these features create a streamlined accessibility tool that empowers visually impaired individuals to manage tasks, retrieve information, and interact with technology more independently and confidently.

\subsection{Overview of Report}
The remainder of this report is structured as follows:

\begin{itemize}
    \item \textbf{Section 2} describes TalkAssist's design and capabilities, including its API architecture, applications, and implementation techniques.
    \item \textbf{Section 3} describes the testing procedure, system performance analysis, and research design.
    \item \textbf{Section 4} addresses security, legal, and ethical issues like user privacy, data security, and dependency on third-party APIs.
    \item \textbf{Section 5} presents a breakdown of the roles and contributions of the team.
    \item \textbf{Section 6} highlights potential future enhancements and growth opportunities for Talk Assist before concluding the paper.
\end{itemize}

%SECTION 2 

\section{Proposed System/Application/Study}

\subsection{Overview}

TalkAssist is designed to support individuals across a broad spectrum of visual impairments. To effectively serve this user group, the system is built around three core requirements: a fully non-visual interface, highly accurate text-to-speech and natural language processing capabilities, and complete hands-free operation through voice interaction. In this section, we provide an overview of our project requirements, describe our system architecture, and walk through the major components of our implementation. We also highlight the core computer science and software engineering principles applied throughout the project including time-complexity considerations, natural language processing techniques, API integration, and system integration techniques.

\subsection{Project Reqiurements}
\subsubsection{User Classes}
There are two distinct user classes. Application Users are individuals who interact directly with TalkAssist. This group includes users with visual impairments who rely on fully voice-driven, non visual interaction. Their primary goal is to communicate with the chatbot, receive accurate spoken responses, and operate the system entirely hands-free. Event Management Users rely on TalkAssist to manage time based tasks such as reminders, alerts, and scheduled notifications. To stay organized without requiring visual involvement, these users rely on precise voice-controlled scheduling and clear audio confirmations.
\subsubsection{Functional Reqiurments}
\begin{itemize}
\item\textbf{Front End GUI:} Talk Assist requires a simple and clear front-end GUI that is able to transcribe conversations and also indicate when either offline or online mode is currently running. The GUI must also contain a reminders page that clearly shows each active reminder and is able to differentiate between active and inactive reminders.

\item\textbf{Online Mode:} Talk Assist requires an online mode that is able to perform commands such as telling the time, weather, date, and setting reminders. It should also be able to search the web for answers to questions asked by the user. We require this online connection as Talk Assist’s online mode has many more capabilities and is meant for at-home use.

\item\textbf{Offline Mode:} Talk Assist requires a lightweight offline mode that is able to perform similar but more basic tasks than its online mode. Offline mode should still be able to tell time, weather, date, and set reminders. There are no web search capabilities, but this is offset by a basic calculation tool that allows offline mode to perform operations requested by the user. Offline mode is lightweight due to this mode being intended for out-of-home use, such as while walking or grocery shopping.

\item\textbf{Wake Up Word \& Hot Key:} Due to Talk Assist’s hands-off approach to operation, we require wake-up word and hotkey capabilities. The wake-up word is meant to activate the program and start a conversation once detected, allowing the user to avoid restarting the program to begin a new conversation. The hotkey is required to allow the user to launch the application and begin microphone recording for wake-up word activation, enabling them to avoid navigating a screen and streamlining the application launch process.
\end{itemize}
\subsubsection{Non-Functional Reqiurements}
\begin{itemize}
\item\textbf{Accessibility:}The system must be fully usable without visual navigation (screen-reader friendly, voice-first interaction)
\item\textbf{Usability:}The chatbot must have a simple, natural, and intuitive voice-based interaction flow. Responses should be clear, concise, and easy to understand for users with varying levels of technical skill
\item\textbf{Scalability:}App should support integration with additional features ( calendars, weather APIs, reminders) without major redesign
\item\textbf{Portability:}Should scale to run on multiple platforms in future (desktop and mobile) with minimal configuration
\end{itemize}

\end{document}
